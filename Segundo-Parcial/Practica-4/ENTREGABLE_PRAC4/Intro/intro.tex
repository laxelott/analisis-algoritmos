
\chapter{Introducción}



% Resumen
\section{Resumen}
    La práctica consta de implementar un algoritmo que pueda descomponer en matrices una imagen y con estas matrices rotar la imagen. La técnica a utilizar es \textbf{Divide y Vencerás} que será empleada en el análisis del algoritmo. El algoritmo fue codificado en el lenguaje de programación \textbf{Python} ejecutado en un entorno de desarrollo \textbf{Linux}.
    
    \textbf{Palabras Clave}: Python, Algoritmo, Dividir, Matriz.

% La introducción a la práctica
\section{Introducción}

    Los algoritmos son una parte fundamental de la ciencia de la computación, ya que estos al ser computables pueden dar solución o una idea más concreta acerca de la solución de un problema.
    
    Un algoritmo no siempre dará una solución correcta, lo cual jamás será malo, porque esto nos ayudará a poder minimizar su radio de error. Una característica casi obligatoria para el buen funcionamiento de un algoritmo es su \textbf{rendimiento y eficacia}. El rendimiento adecuado se encuentra en la solución más rápida y menos costosa \cite{Algorithm}.

    Una forma de poder atacar algoritmos complejos es por la técnica de Divide y Vencerás. Consiste en resolver la complejidad de un algoritmo mediante una resolución recursiva dividiendo el problema en dos o más subproblemas, repitiendo ese paso hasta que los subproblemas se vuelvan tan sencillos para que se puedan resolver directamente \cite{DivideYVencerass}.

    Dicha técnica se utilizará para encontrar la complejidad del algoritmo realizado en esta práctica. El algoritmo de Rotación de Imagen. Una imagen puede ser descompuesta píxel a píxel, esto gracias a que un píxel RGB posee una matriz de valores (R,G,B), el propósito es almacenar y rotar estas matrices haciendo inspiración del algoritmo de Strassen. 
    Dentro de la practica también se anexan ejercicios que resuelven problemas de complejidad.


    %

    
