
\chapter{Introducción}



% Resumen
\section{Resumen}
    La práctica consta de implementar un algoritmo que pemita comparar dos archivos de texto plano mediante \textbf{Programación Dinámica} haciendo uso del algoritmo de la subsecuencia común más larga. Siendo desarrollada en un ambiente de programación con \textbf{Python} y \textbf{Linux}.
    
    \textbf{Palabras Clave}: Python, Programación Dinámica, Recursividad.

% La introducción a la práctica
\section{Introducción}

    Los algoritmos son una parte fundamental de la ciencia de la computación, ya que estos al ser computables pueden dar solución o una idea más concreta acerca de la solución de un problema.
    
    Un algoritmo no siempre dará una solución correcta, lo cual jamás será malo, porque esto nos ayudará a poder minimizar su radio de error. Una característica casi obligatoria para el buen funcionamiento de un algoritmo es su \textbf{rendimiento y eficacia}. El rendimiento adecuado se encuentra en la solución más rápida y menos costosa \cite{Algorithm}.

    Un problema desarrolla complicaciones debido al tiempo y el uso de sus recursos, una alternativa para poder amplificar soluciones es la \texbf{Programación Dinámica} que se inspira del camino más corto dentro de una solución óptima utilizando técnicas como la subdivisión de problemas reduciendo de esta manera el tiempo de ejecución \cite{Dinamica}. En esta práctica se analiza principalmente este paradigma mientras se analiza un algoritmo que permita el análisis de la subsecuencia común más larga de dos archivos planos de texto que contienen cadenas de texto. 


    %

    
