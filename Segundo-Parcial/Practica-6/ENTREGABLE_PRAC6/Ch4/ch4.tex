\chapter{Conclusiones}

\section{Conclusiones Generales}
    
    Pese a ser una práctica sencilla, tuvimos problemas dentro del análisis a Priori, siendo que no entendimos de forma correcta el analizar un algoritmo que entre en el paradigma de la Programación Dinámica, creemos que es fue nuestro único punto del cuál tuvimos más problemas. La teoría de la Programación Dinámica es eficiente para poder tratar con problemas que no se encuentra una solución rápida. Nos resultó de ayuda el algoritmo de la subsecuencia común más larga para poder entender su funcionamiento.

\newpage
\section{Isaac Sánchez - Conclusiones}
    El algoritmo de la subsecuencia común más larga me pareció de forma inteligente una buena introducción para el desarrollo de soluciones que requieran una eficacia mejor. Mi problema surgió en el análisis del algoritmo, que fue resuelto después de verificar diferentes fuentes en las cuales me apoye para dar una conclusión más certera a mi análisis. 
    \begin{figure}[htp!]
            \centering
            \includegraphics[width=1 \textwidth]{Images/Fotos_Alumnos/274612600_2528992867236334_6677874837890685705_n.jpg}  
            \caption{Isaac Sánchez}
            \label{fig:my_label1}
        \end{figure}
    


\newpage
\section{Axel Trevino - Conclusiones}
    No veo diferencias entre divide y vencerás y programación dinámica, pero al menos tiene un nombre bonito
    \begin{figure}[htp!]
            \centering
            \includegraphics[width=0.4 \textwidth]{Images/Fotos_Alumnos/axel.jpg}  
            \caption{Axel Treviño}
            \label{fig:my_label2}
        \end{figure}
