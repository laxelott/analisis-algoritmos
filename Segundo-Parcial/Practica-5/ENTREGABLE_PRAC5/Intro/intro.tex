
\chapter{Introducción}



% Resumen
\section{Resumen}
    La práctica consta de implementar un algoritmo que resuelva un problema de un granjero que busca optimizar sus rutas de desplazamiento hacia el pueblo donde tiene que conseguir su fertilizante haciendo uso del \textbf{algoritmo de Greedy}. Dicha práctica será desarrollada en un ambiente de programación con \textbf{Python} y \textbf{Linux}. 
    
    \textbf{Palabras Clave}: Python, Algoritmo, Greedy.

% La introducción a la práctica
\section{Introducción}

    Los algoritmos son una parte fundamental de la ciencia de la computación, ya que estos al ser computables pueden dar solución o una idea más concreta acerca de la solución de un problema.
    
    Un algoritmo no siempre dará una solución correcta, lo cual jamás será malo, porque esto nos ayudará a poder minimizar su radio de error. Una característica casi obligatoria para el buen funcionamiento de un algoritmo es su \textbf{rendimiento y eficacia}. El rendimiento adecuado se encuentra en la solución más rápida y menos costosa \cite{Algorithm}.

    Una solución a problemas que tengan que ver con encontrar una solución óptima, es hacer el planteamiento del Algoritmo de Greedy. Este algoritmo es voraz y trabaja para solucionar problemas de optimización. Su fuerte es que es un algoritmo sencillo de implementar, siendo rápido a la hora de brindar soluciones óptima \cite{Greedy}. La práctica hace uso del algoritmo para solucionar el problema del granjero que necesita saber qué días tiene que abastecer su suministro de fertilizante para que no se agote. 


    %

    
