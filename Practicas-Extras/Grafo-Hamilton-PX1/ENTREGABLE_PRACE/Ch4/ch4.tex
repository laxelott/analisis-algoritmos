\chapter{Conclusiones}

\section{Conclusiones Generales}
    
    Siendo la practica extra del semestre, nos esmeramos en poder demostrar que la practica es posible. Concluimos que el algoritmo de los grafos es un ejercicio que si bien resuelve un grafo puede ser utilizado dentro de otras practicas en la vida cotidiana, hasta para el trafico automovilistico en una ciudad. La complejidad no resulto tan pesada y obtuvimos resultados esperados.

\newpage
\section{Isaac Sánchez - Conclusiones}
    El algoritmo de Hamilton me pareció de forma inteligente una buena introducción para el análisis de recorridos dentro de un problema, evitando lugares que ya se han visitado. Tuve complicaciones al pensar en el análisis a priori, ya que no identificaba de manera correcta el grado de complejidad de los ciclos que tomamos para la solucion. 
    \begin{figure}[htp!]
            \centering
            \includegraphics[width=1 \textwidth]{Images/Fotos_Alumnos/274612600_2528992867236334_6677874837890685705_n.jpg}  
            \caption{Isaac Sánchez}
            \label{fig:my_label1}
        \end{figure}
    


\newpage
\section{Axel Trevino - Conclusiones}
    Fue una practica sencilla, ya que los conocimientos aplicados a lo largo del curso pudieron ser resumidos en solamente un algoritmo. Presente una complicación a la hora de comprender que era lo que el algoritmo debía impirimir. 
    \begin{figure}[htp!]
            \centering
            \includegraphics[width=0.4 \textwidth]{Images/Fotos_Alumnos/axel.jpg}  
            \caption{Axel Treviño}
            \label{fig:my_label2}
        \end{figure}
