
\chapter{Introducción}



% Resumen
\section{Resumen}
    La práctica consta de implementar un algoritmo que permita encontrar si un grafo pertenece a un grafo hamiltoniano. Siendo desarrollada en un ambiente de programación con \textbf{Python} y \textbf{Linux}.
    
    \textbf{Palabras Clave}: Python, Grafos, Hamiltoniano.

% La introducción a la práctica
\section{Introducción}

    Los algoritmos son una parte fundamental de la ciencia de la computación, ya que estos al ser computables pueden dar solución o una idea más concreta acerca de la solución de un problema.
    
    Un algoritmo no siempre dará una solución correcta, lo cual jamás será malo, porque esto nos ayudará a poder minimizar su radio de error. Una característica casi obligatoria para el buen funcionamiento de un algoritmo es su \textbf{rendimiento y eficacia}. El rendimiento adecuado se encuentra en la solución más rápida y menos costosa \cite{Algorithm}.

    Un camino \textbf{hamiltoniano} en un grafo es un camino (es decir, una sucesión de aristas adyacentes), que visita todos los vértices del grafo una sola vez. Si además el primer y último vértice visitado coincide, el camino es un ciclo hamiltoniano. No obstante, los ciclos y caminos actualmente denominados hamiltonianos aparecieron mucho antes. Al parecer, ya en el siglo IX el poeta indio Rudrata nombra el llamado camino del caballo. Se trata de una sucesión de movimientos del caballo sobre un arcidriche de manera que esta pieza, el caballo, visite todos y cada uno de los escaques una sola vez. Se trata, en consecuencia, de encontrar un camino hamiltoniano en un grafo cuyos vértices son los escaques de un arcidriche de manera que dos vértices son adyacentes si y sólo si se puede pasar de uno a otro mediante un movimiento de caballo \cite{Hamil}. 




    %

    
