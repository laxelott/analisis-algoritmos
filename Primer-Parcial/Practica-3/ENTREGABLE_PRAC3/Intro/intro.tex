
\chapter{Introducción}



% Resumen
\section{Resumen}
    La practica consta del análisis a priori y posteriori de cinco algoritmos, 3 iterativos y 2 recursivos. Ambos algoritmos fueron desarrollados mediante el lenguaje de programación \textbf{Python} en un ambiente virtual de \textbf{Linux}.

    \textbf{Palabras Clave}: Python, Algoritmo, Recursividad, Iteratividad, Complejidad. 

% La introducción a la práctica
\section{Introducción}
    Los algoritmos son una parte fundamental de la ciencia de la computación, ya que estos al ser computables pueden dar solución o una idea más concreta acerca de la solución de un problema.
    
    Un algoritmo no siempre dará una solución correcta, lo cual jamás será malo, porque esto nos ayudará a poder minimizar su radio de error. Una característica casi obligatoria para el buen funcionamiento de un algoritmo es su \textbf{rendimiento y eficacia}. El rendimiento adecuado se encuentra en la solución más rápida y menos costosa \cite{Algorithm}.
    
    Los algoritmos presentados en esta práctica fueron desarrollados siguiendo los principios de iteratividad y recursividad para poder obtener los resultados esperados planteados en cada problema. Los primeros algoritmos, iterativos y recursivos, presentados definen el cálculo de cocientes, mostrando el resultado de una división de enteros positivos sin sus residuos. 
    El último algoritmo a comprende una búsqueda terciaria dentro de un arreglo, el arreglo es partido en 3 subarreglos para facilitar la búsqueda entre cada arreglo. El resultado es arrojar un \textit{booleano} que indique si el elemento buscado está dentro de alguno de los arreglos.
