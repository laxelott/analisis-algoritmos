
\chapter{Introducción}



% Resumen
\section{Resumen}
    La práctica consta del análisis de complejidad de dos algoritmos iterativos y sus respectivas gráficas.

    \begin{description}
        \item [Python]
        \item [Algoritmo]
        \item [Complejidad]
    \end{description}

% La introducción a la práctica
\section{Introducción}
    Los algoritmos son una parte fundamental de la ciencia de la computación, ya que estos al ser computables pueden dar solución o una idea más concreta acerca de la solución de un problema.
    
    Un algoritmo no siempre dará una solución correcta, lo cual jamás será malo, porque esto nos ayudará a poder minimizar su radio de error. Una característica casi obligatoria para el buen funcionamiento de un algoritmo es su \textbf{rendimiento y eficacia}. El rendimiento adecuado se encuentra en la solución más rápida y menos costosa. \cite{Algorithm}
    
    La importancia de conocer el costo y las soluciones de un algoritmo es que sepamos si las respuestas dadas son las esperadas. Esto nos ayuda a resolver los problemas de manera concisa. (cita)
    
    Dentro de esta práctica, podremos ver dos algoritmos puestos a prueba de su complejidad temporal. Dando respuesta a sus diversas soluciones, desde las más eficientes hasta las que hacen que el algoritmo no pueda proporcionar las respuestas esperadas. 
