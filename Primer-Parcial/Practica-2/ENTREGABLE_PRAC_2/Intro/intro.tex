
\chapter{Introducción}



% Resumen
\section{Resumen}
    La practica consta del análisis a priori y posteriori de dos algoritmos; sucesión de Fibonacci y encontrar números perfectos. Ambos algoritmos fueron desarrollados mediante el lenguaje de programación \textbf{Python} en un ambiente virtual de \textbf{Linux}.

    \textbf{Palabras Clave}: Python, Algoritmo, Temporalidad, Análisis.

% La introducción a la práctica
\section{Introducción}
    Los algoritmos son una parte fundamental de la ciencia de la computación, ya que estos al ser computables pueden dar solución o una idea más concreta acerca de la solución de un problema.
    
    Un algoritmo no siempre dará una solución correcta, lo cual jamás será malo, porque esto nos ayudará a poder minimizar su radio de error. Una característica casi obligatoria para el buen funcionamiento de un algoritmo es su \textbf{rendimiento y eficacia}. El rendimiento adecuado se encuentra en la solución más rápida y menos costosa \cite{Algorithm}.
    
    Los algoritmos presentados en esta práctica presentan una complejidad temporal polinomial y no polinomial, siendo calculados por cuanto tarda en ejecutarse dependiendo la cantidad de valores de entrada \cite{CTemp}.
